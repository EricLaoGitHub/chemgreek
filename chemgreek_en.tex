% arara: pdflatex: { shell: on }
% arara: biber
% arara: pdflatex
% arara: pdflatex
% --------------------------------------------------------------------------
% the CHEMMACROS bundle
%   chemgreek_en.tex
% macros and commands for chemists
% --------------------------------------------------------------------------
% Clemens Niederberger
% --------------------------------------------------------------------------
% https://bitbucket.org/cgnieder/chemmacros/
% contact@mychemistry.eu
% --------------------------------------------------------------------------
% If you have any ideas, questions, suggestions or bugs to report, please
% feel free to contact me.
% --------------------------------------------------------------------------
% Copyright 2011-2013 Clemens Niederberger
%
% This work may be distributed and/or modified under the
% conditions of the LaTeX Project Public License, either version 1.3
% of this license or (at your option) any later version.
% The latest version of this license is in
%   http://www.latex-project.org/lppl.txt
% and version 1.3 or later is part of all distributions of LaTeX
% version 2005/12/01 or later.
%
% This work has the LPPL maintenance status `maintained'.
%
% The Current Maintainer of this work is Clemens Niederberger.
% --------------------------------------------------------------------------
\documentclass[load-preamble+]{cnltx-doc}
\usepackage[utf8]{inputenc}
\usepackage[greek=newtx]{chemmacros}
\setcnltx{
  package  = {chemgreek},
  title    = \huge the \chemmacros\ bundle,
  url      = https://bitbucket.org/cgnieder/chemmacros/ ,
  authors  = Clemens Niederberger ,
  email    = contact@mychemistry.eu ,
  info     = {
    packages \chemmacros\ (v\csname chemmacros@version\endcsname),
    \chemformula\ (v\csname chemformula@version\endcsname),
    \ghsystem\ (v\csname ghsystem@version\endcsname) and
    \chemgreek\ (v\csname chemgreek@version\endcsname)\\[2ex]
    {\Large documentation for the \chemgreek\ package}} ,
  abstract = {%
    \centering
    \includegraphics{chemmacros-logo.pdf}
    \par
  } ,
  add-cmds = {
     abinitio, activatechemgreekmapping, AddRxnDesc, anti, aq, aqi,
     ba, bond, bridge,
    cd, ch, changechemgreeksymbol, charrow, chcpd, chemabove, chemalpha,
      chembeta, chemgamma, chemdelta, chemDelta, chemformula@bondlength,
      chemomega, chemphi, chemPhi, chemsetup, chlewis, chname , cip, cis, ch,
      CNMR,
    data, DeclareChemArrow, DeclareChemBond, DeclareChemBondAlias,
      declarechemgreekmapping, DeclareChemIUPAC, DeclareChemLatin,
      DeclareChemNMR, DeclareChemParticle, DeclareChemPhase,
      DeclareChemReaction, DeclareChemState, delm, delp, Delta, Dfi,
    el, ElPot, endo, Enthalpy, enthalpy, Entropy,
    fmch, fpch, fscrm, fscrp,
    gas, ghs, ghslistall, ghspic, Gibbs, gram,
    hapto, HNMR, Helmholtz, hydrogen,
    insitu, invacuo, iupac,
    Ka, Kb, Kw,
    Lfi, listofreactions, lqd,
    mch, mega, meta, mhName,
    NewChemArrow, NewChemBond, NewChemBondAlias,
      newchemgreekmapping, NewChemIUPAC, NewChemLatin,
      NewChemNMR, NewChemParticle, NewChemPhase,
      NewChemReaction, NewChemState,
      newman, nitrogen, NMR, Nu, Nuc,
    orbital, ortho, ox, OX, oxygen,
    para, pch, per, pH, phase, phosphorus, photon, pKa, pKb, pOH, pos,
      positron, Pot, prt, printchemgreekalphabet ,
    Rad, redox, RenewChemArrow, RenewChemBond, renewchemgreekmapping,
      RenewChemIUPAC, RenewChemLatin, RenewChemNMR, RenewChemParticle,
      RenewChemPhase, RenewChemState,
    Sf, scrm, scrp, second, selectchemgreekmapping, setchemformula,
      ShowChemArrow, ShowChemBond, sld, Sod, State, sulfur,
    trans,
    val
  } ,
  add-silent-cmds = {
    addplot,
    bottomrule,
    cancel, cdot, ce, cee, celsius, centering, chemfig, chemname, clap,
      cnsetup, color, cstack, cstsetup,
    DeclareInstance, DeclareSIUnit, definecolor, draw,
    electronvolt,
    footnotesize,
    glqq, grqq,
    hertz, hspace,
    includegraphics, intertext, IUPAC,
    joule,
    kilo,
    latin, lewis, Lewis, liquid, ltn,
    metre, midrule, milli, mmHg, mole,
    nano, nicefrac, num, numrange,
    ominus, oplus,
    percent, pgfarrowsdeclarealias, pgfarrowsrenewalias,
    renewtagform, rightarrow,
    sample, scriptscriptstyle, setatomsep, setbondoffset, setmainfont, sfrac,
      shorthandoff, si, SI, sisetup, square, subsection,
    textcolor, textendash, textsuperscript, tiny, toprule,
    upbeta, upeta, upgamma, upPhi, upphi, usetikzlibrary,
    volt, vphantom, vspave,
    xspace,
    z@, z@skip
  }
}

\usepackage{chemfig,booktabs,cancel,varioref}
\usepackage[version=3]{mhchem}

\expandafter\def\csname libertine@figurestyle\endcsname{LF}
\usepackage[libertine]{newtxmath}
\expandafter\def\csname libertine@figurestyle\endcsname{OsF}

\usepackage[biblatex]{embrac}
\ChangeEmph{[}[,.02em]{]}[.055em,-.08em]
\ChangeEmph{(}[-.01em,.04em]{)}[.04em,-.05em]

\usepackage[accsupp]{acro}
\acsetup{
  long-format  = \scshape ,
  short-format = \scshape
}
\DeclareAcronym{ghs}{
  short     = ghs ,
  long      = Globally Harmonized System of Classification and Labelling of
    Chemicals ,
  pdfstring = GHS ,
  accsupp   = GHS
}
\DeclareAcronym{eu}{
  short     = EU ,
  long      = European Union ,
  pdfstring = EU ,
  accsupp   = EU
}
\DeclareAcronym{iupac}{
  short     = iupac ,
  long      = International Union of Pure and Applied Chemistry ,
  pdfstring = IUPAC ,
  accsupp   = IUPAC
}
\DeclareAcronym{UN}{
  short     = un ,
  long      = United Nations ,
  pdfstring = UN ,
  accsupp   = UN
}
\DeclareAcronym{dvi}{
  short     = dvi ,
  long      = device independent file format ,
  pdfstring = DVI ,
  accsupp   = DVO
}
\DeclareAcronym{pdf}{
  short     = pdf ,
  long      = portable document file ,
  pdfstring = PDF ,
  accsupp   = PDF
}

\chemsetup{
  option/synchronize ,
  chemformula/format = \libertineLF
}
% \colorlet{chemformula}{black!90}

\sisetup{
  detect-mode=false,
  mode=text,
  text-rm=\libertineLF
}

\usepackage{filecontents}

\defbibheading{bibliography}{\addsec{References}}
\addbibresource{\jobname.bib}
\begin{filecontents*}{\jobname.bib}
@book{iupac:greenbook,
  author    = {E. Richard Cohan and Tomislav Cvita\v{s} and Jeremy G. Frey and
    Bertil Holmstr\"om and Kozo Kuchitsu and Roberto Marquardt and Ian Mills and
    Franco Pavese and Martin Quack and J\"urgen Stohner and Herbert L. Strauss and
    Michio Takami and Anders J Thor} ,
  title     = {``Quantities, Symbols and Units in Physical Chemistry'', \acs{iupac}
    Green Book} ,
  sorttitle = {Quantities, Symbols and Units in Physical Chemistry} ,
  indexsorttitle = {Quantities, Symbols and Units in Physical Chemistry} ,
  edition   = {3rd Edition. 2nd Printing} ,
  year      = {2008} ,
  publisher = {\acs{iupac} \&\ RSC Publishing, Cambridge}
}
@book{iupac:redbook,
  author    = {Neil G. Connelly and Ture Damhus and Richard M. Hartshorn and
    Alan T. Hutton} ,
  title     = {``Nomenclature of Inorganic Chemistry'', \acs{iupac} Red Book} ,
  sorttitle = {Nomenclature of Inorganic Chemistry} ,
  indexsorttitle = {Nomenclature of Inorganic Chemistry} ,
  year      = {2005} ,
  publisher = { \acs{iupac} \&\ RSC Publishing, Cambridge} ,
  isbn      = {0-85404-438-8}
}
@misc{eu:ghsystem_regulation,
  author   = {{The European Parliament and The Council of the European Union}},
  title    = {Regulation (EC) No 1272/2008 of the European Parliament and of
    the Council} ,
  subtitle = {on classification, labelling and packaging of substances and
    mixtures, amending and repealing Directives 67/548/EEC and 1999/45/EC, and
    amending Regulation (EC) No 1907/2006} ,
  journal  = {Official Journal of the European Union} ,
  date     = {2008-12-16}
}
@online{unece:ghsystem_implementation,
  author   = {United Nations Economic Commission for Europe} ,
  title    = {GHS Implementation} ,
  url      =
    {http://www.unece.org/trans/danger/publi/ghs/implementation_e.html} ,
  urldate  = {2012-03-20} ,
  date     = {2012-03-20}
}
\end{filecontents*}

\DeclareInstance{xfrac}{chemformula-text-frac}{text}
  {
    scale-factor        = 1 ,
    denominator-bot-sep = -.2ex ,
    denominator-format  = \scriptsize #1 ,
    numerator-top-sep   = -.2ex ,
    numerator-format    = \scriptsize #1 ,
    slash-right-kern    = .05em ,
    slash-left-kern     = .05em
  }

\usetikzlibrary{calc,positioning,decorations.pathmorphing,patterns}

\newpackagename\chemmacros{chemmacros}
\newpackagename\chemformula{chemformula}
\newpackagename\ghsystem{ghsystem}
% \newpackagename\chemgreek{chemgreek}


\newidxcmd\manual{\textsf{#1}}[\ (manual)]

\newenvironment{codedesc}
  {%
    \def\Code##1{\item\code{##1}\hfill\newline}%
    \cnltxlist
  }
  {\endcnltxlist}

\renewcommand*\AmS{\hologo{AmS}}

\newcommand*\TikZ{Ti\textit{k}Z}

\newcommand*\tablehead[1]{\textrm{\bfseries#1}}

\begin{document}

\section{Introduction}
The \chemgreek\ package is an auxiliary package for other chemistry packages
such as \chemmacros.  In chemistry there is often the need for upright greek
letters.  The \chemgreek\ package provides an interface to various other
packages that provide upright greek letters.  One could mention
\pkg{textgreek}, \pkg{upgreek}, \pkg{newtx} or \pkg{kpfonts}.  All of these
packages provide upright greek letters, some a whole alphabet some only the
upright variants of the standard italic symbols for which macros are defined
in base \LaTeX.

\chemgreek\ offers a possibility to map those different interfaces to a
unified set of macros for usage in a chemistry package.  This is useful as
then for example names like \iupac{\b\-\D\-gluco\|pyranose} can be typeset
with a semantic interface and still have matching greek letters while the user
is not limited to a certain package or font.  Consequently this package is
used by \chemmacros\ and its \acs{iupac} naming commands.

\section{Licence and Requirements}
\license

\chemgreek\ loads the following packages:
\pkg{expl3}\footnote{\CTANurl{l3kernel}}~\cite{bnd:l3kernel} and
\pkg{xparse}\footnote{\CTANurl{l3packages}}~\cite{bnd:l3packages}.

\section{News}
\subsection{Version~0.2}
\begin{itemize}
  \item The mapping ``mathdesign'' has been added.  In order to use it you
    need the \pkg{mathdesign} package~\cite{pkg:mathdesign} loaded.
  \item The mapping ``fourier'' has been added.  In order to use it you
    need the \pkg{fourier} package~\cite{pkg:fourier} loaded.
\end{itemize}

\subsection{Version~0.3}
\begin{itemize}
  \item The provided macros have been renamed from \cs*{Chem\meta{\ldots}}
    into \cs*{chem\meta{\ldots}}.  The uppercase version still are provided
    for backwards compatibility but issue a warning message and will be
    removed some time in the future.
  \item The commands for defining mappings have gotten an optional argument
    which allows to specify the name of the package a mapping needs.  The
    command \cs{selectchemgreekmapping} now checks for this package and gives
    a warning if it doesn't find it loaded.
  \item The mapping ``textalpha'' has been added.  In order to use it you
    need the \pkg{textalpha} package (part of
    \bnd{greek-fontenc}~\cite{bnd:greek-fontenc}) loaded.
  \item If the package \pkg{hyperref}~\cite{pkg:hyperref} is loaded with the
    \code{unicode} option \emph{and} the \pkg{textalpha} package has been
    loaded at begin document all the \cs*{chem\meta{\ldots}} commands are let
    to \pkg{textalpha}'s \cs*{text\meta{\ldots}} commands for the \acs{pdf}
    bookmarks.  This allows Greek letters in the bookmarks without worrying
    about \cs*{texorpdfstring}.
\end{itemize}

\subsection{Version~0.4}
\begin{itemize}
  \item The mapping ``fontspec'' has been added.  In order to use it you
    need the \pkg{fontspec} package~\cite{pkg:fontspec} loaded.  This means it
    can only be used with \LuaLaTeX\ or \XeLaTeX.
  \item New command \cs{printchemgreekalphabet}.
\end{itemize}

\section{Define Mappings}
\selectchemgreekmapping{default}

\chemgreek's main commands are:
\begin{commands}
  \command{newchemgreekmapping}[\oarg{package}\marg{name}\marg{mapping list}]
    \changedversion{0.3}Add a new mapping to \chemgreek.  Issue an error if it
    already exists.  With the optional argument the package that is needed for
    this mapping can (and should) be specified.
  \command{renewchemgreekmapping}[\oarg{package}\marg{name}\marg{mapping list}]
    \changedversion{0.3}Renew a \chemgreek\ mapping.  Issue an error if it
    doesn't exist yet.  With the optional argument the package that is needed
    for this mapping can (and should) be specified.
  \command{declarechemgreekmapping}[\oarg{package}\marg{name}\marg{mapping list}]
    \changedversion{0.3}Declare a new mapping to \chemgreek.  If the mapping
    already exists it will be overwritten.  With the optional argument the
    package that is needed for this mapping can (and should) be specified.
\end{commands}

The command \cs{newchemgreekmapping} needs to get a comma separated list of
24 pairs divided by a slash.  The first entry is the lowercase version und the
second the uppercase version for the corresponding greek letter at the current
position.  This will become clearer if you look at how the \code{default}
mapping is defined:

\begin{sourcecode}
  \newchemgreekmapping{default}
    {
      \ensuremath{\alpha}   / \ensuremath{\mathrm{A}} , %  1: alpha
      \ensuremath{\beta}    / \ensuremath{\mathrm{B}} , %  2: beta
      \ensuremath{\gamma}   / \ensuremath{\Gamma} ,     %  3: gamma
      \ensuremath{\delta}   / \ensuremath{\Delta} ,     %  4: delta
      \ensuremath{\epsilon} / \ensuremath{\mathrm{E}} , %  5: epsilon
      \ensuremath{\zeta}    / \ensuremath{\mathrm{Z}} , %  6: zeta
      \ensuremath{\eta}     / \ensuremath{\mathrm{H}} , %  7: eta
      \ensuremath{\theta}   / \ensuremath{\Theta} ,     %  8: theta
      \ensuremath{\iota}    / \ensuremath{\mathrm{I}} , %  9: iota
      \ensuremath{\kappa}   / \ensuremath{\mathrm{K}} , % 10: kappa
      \ensuremath{\lambda}  / \ensuremath{\Lambda} ,    % 11: lambda
      \ensuremath{\mu}      / \ensuremath{\mathrm{M}} , % 12: mu
      \ensuremath{\nu}      / \ensuremath{\mathrm{N}} , % 13: nu
      \ensuremath{\xi}      / \ensuremath{\Xi} ,        % 14: xi
      \ensuremath{o}        / \ensuremath{\mathrm{O}} , % 15: omikron
      \ensuremath{\pi}      / \ensuremath{\Pi} ,        % 16: pi
      \ensuremath{\rho}     / \ensuremath{\mathrm{P}} , % 17: rho
      \ensuremath{\sigma}   / \ensuremath{\Sigma} ,     % 18: sigma
      \ensuremath{\tau}     / \ensuremath{\mathrm{T}} , % 19: tau
      \ensuremath{\upsilon} / \ensuremath{\Upsilon} ,   % 20: upsilon
      \ensuremath{\phi}     / \ensuremath{\Phi} ,       % 21: phi
      \ensuremath{\psi}     / \ensuremath{\Psi} ,       % 22: psi
      \ensuremath{\chi}     / \ensuremath{\mathrm{X}} , % 23: chi
      \ensuremath{\omega}   / \ensuremath{\Omega}       % 24: omega
  }
\end{sourcecode}

There \emph{must} be 24 pairs of entries, \ie, a complete mapping!  Those
entries are the ones that will be used by the interface macros.  For each
letter a pair \cs{chemalpha}/\cs{chemAlpha} is defined that uses the entries
of the currently active mapping.  That means there are 48 (robust) macros
defined each beginning with \cs*{chem...} followed by the lowercase or
uppercase name of the Greek letter.

The default mapping is -- as you can probably see -- \emph{not an upright
  one}.  This is because \chemgreek\ will not make any choice for a specific
package but let's the user (or another package) choose.  The \chemmacros\
package for example provides a package option that selects one of the
available mappings.

\begin{example}
  Default mapping: \chemphi\ and \chemPhi, $\phi$ and $\Phi$
\end{example}

\section{Predefined Mappings and Selection of a Mapping}
\chemgreek\ predefines some mappings.  Some of the mappings require additional
packages to be loaded.  The mapping names and the required packages are listed
in table~\ref{tab:mappings}.  Ths mapping \code{fontspec} is a bit different
here:  if you use this mapping then the fact is used that \pkg{fontspec} also
defines commands like \cs*{textalpha}.  However, they only work if you also
use a font that has the Greek glyphs.

\begin{table}
  \centering
  \begin{tabular}{>{\ttfamily}ll}
    \toprule
      \tablehead{mapping} & \tablehead{package} \\
    \midrule
      default     & --- \\
      var-default & --- \\
      textgreek   & \pkg{textgreek} \cite{pkg:textgreek} \\
      upgreek     & \pkg{upgreek} \cite{pkg:upgreek} \\
      newtx       & \pkg*{newtxmath} \cite{pkg:newtx} \\
      kpfonts     & \pkg{kpfonts} \cite{pkg:kpfonts} \\
      mathdesign  & \pkg{mathdesign} \cite{pkg:mathdesign} \\
      fourier     & \pkg{fourier} \cite{pkg:fourier} \\
      textalpha   & \pkg{textalpha} \cite{bnd:greek-fontenc} \\
      fontspec    & \pkg{fontspec} \cite{pkg:fontspec} \\
    \bottomrule
  \end{tabular}
  \caption{Predefined mappings.}
  \label{tab:mappings}
\end{table}

A mapping is selected and activated with one of the following commands:
\begin{commands}
  \command{activatechemgreekmapping}[\sarg\marg{name}]
    \changedversion{0.3}This commands selects and activates the mapping
    \meta{name}.  If the star variant is used also the package of mapping
    \meta{name} (as defined with \cs{newchemgreekmapping} is loaded.  The
    command can only be used in the document preamble.
  \command{selectchemgreekmapping}[\marg{name}]
    \changedversion{0.3}This commands selects and activates the mapping
    \meta{name}.  A required package has to be loaded additionally the usual
    way via \cs*{usepackage} or \cs*{RequirePackage}.  If the package hasn't
    been loaded a warning will be written to the log.  The command can be used
    throughout the document.
\end{commands}

\begin{example}
  % requires the `newtxmath' package to be loaded:
  \chemphi\ and \chemPhi, $\phi$ and $\Phi$\par
  \selectchemgreekmapping{newtx}
  \chemphi\ and \chemPhi, $\upphi$ and $\upPhi$
\end{example}

Since the \code{fontspec} mapping is a little bit different than the others
I'd like to show a little example for it.  The difference is subtle: you need
to choose a font containing the needed glyphs.

\begin{example}[compile,program=lualatex,runs=1,add-frame=false]
  \documentclass[margin=3pt]{standalone}
  \usepackage{fontspec}
  \setmainfont{Linux Libertine O}% need a font that has the glyphs!
  \usepackage{chemgreek}
  \selectchemgreekmapping{fontspec}
  \begin{document}
  \printchemgreekalphabet
  \end{document}
\end{example}

\section{Changing a Specific Symbol in an Existing Mapping}
If you should want to change a specific entry of a specific mapping it would
be rather tedious to redefine the whole mapping.  That is why \chemgreek\
provides a command for that purpose:
\begin{commands}
  \command{changechemgreeksymbol}[\marg{mapping
    name}\Marg{upper|lower}\marg{entry name}\marg{entry}]
    Changes the \code{upper}- or \code{lower}case entry \meta{entry name} in
    the mapping \meta{mapping name}.
\end{commands}

In order to activate the change you need the (re-) activate the affected
mapping afterwards:
\begin{example}
  \chemalpha
  \changechemgreeksymbol{default}{lower}{alpha}{xxx}%
  \selectchemgreekmapping{default}
  \chemalpha
\end{example}

\section{Inspecting a Mapping}
\selectchemgreekmapping{newtx}
If you want to check if a mapping has been correctly set you can use the
following commands:
\begin{commands}
  \command{printchemgreekmapping}[\marg{mapping}]
    \sinceversion{0.3}This will typeset a table (using a simple \code{tabular}
    environment) with all~48 characters like the one shown in
    table~\ref{tab:showmapping}.
  \command{printchemgreekalphabet}
    \sinceversion{0.4}This will print the twentyfour pairs of lower- and
    uppercase letters of the currently active mapping: \printchemgreekalphabet.
  \command{showchemgreekmapping}[\marg{mapping}]
    \changedversion{0.3}This command will write information about the
    definition of all 48~macros for a mapping to the log file.
\end{commands}

\begin{table}
  \centering
  \printchemgreekmapping{newtx}
  \caption{A demonstration of the \cs*{printchemgreekmapping} command.}
  \label{tab:showmapping}
\end{table}

\clearpage

\end{document}
